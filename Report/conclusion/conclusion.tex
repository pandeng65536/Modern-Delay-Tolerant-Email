\chapter{Conclusions and Future Work}
\section{Conclusion}

This dissertation has completed a simulation and analysis of a modern delay-tolerant email proposal.  
In the implementation phase, a simulation platform was established based on Docker containerization, integrating a mail server (Postfix/Dovecot), a client (Thunderbird), a DTN protocol stack (ION-DTN), an email gateway (BPMail), and a network simulator (\texttt{tc}).

During the experimental testing phase, a key ``user interplanetary mobility'' roaming scenario was designed and validated.  
The test results revealed a architectural flaw in the proposal: when a user roams to a remote network and attempts to access their home mail server, the real-time synchronization protocols (IMAP/SMTP) are forced to operate over a long-latency link.
This leads to a sharp performance degradation that renders the system unusable, demonstrating that the Johnson Draft does not support user mobility.  

Furthermore, functional testing identified an implementation deficiency in the BPMail gateway component, which was unable to process large attachments, thereby limiting its practical value.

To address the mobility issue, this research proposed a conceptual extension based on an \textit{Sync Agent}.  
This extension decouples the user's real-time interaction with the email system from background data synchronization across interplanetary links by establishing a local agent, offering a potential solution to this limitation.

\section{Limitations}

Although this research achieved its primary objectives, it is subject to the following limitations:

\begin{itemize}
    \item \textit{Simplified Network Environment Simulation}: The simulation primarily focused on long delay, the most prominent characteristic of deep-space networks. Other complex network conditions, such as bit error rates, asymmetric bandwidth, and periodic disruptions, were not extensively tested in combination.
    \item \textit{Lack of DNS and Security Mechanism Simulation}: To focus on the core issues of mail transport and synchronization, the implementation simplified the DNS server configuration. Consequently, it was not possible to experimentally validate the potential failure of DNS-dependent email trust mechanisms (e.g., SPF, DKIM) in the roaming scenario.
    \item \textit{Unverified Nature of the Proposed Extension}: The \textit{On-demand Sync Agent} extension presented in this dissertation remains at the conceptual design stage. Its effectiveness, and any new issues it might introduce (such as implementation difficulties) have not been verified through prototype implementation.
\end{itemize}

\section{Future Work}

Based on the findings and limitations of this research, several avenues for future work can be identified:

\begin{itemize}
    \item \textit{Improving DTN Gateway Support for Large Files}: The most direct and practically valuable line of future work would be to address the shortcomings of BPMail. This could involve debugging and patching its source code or developing a new, more robust mail gateway to ensure reliable handling of large attachments, thereby removing a key barrier to the proposal's practical deployment.
    \item \textit{Implementation and Validation of the Proposed Solution}: The next logical step is to transform the \textit{On-demand Sync Agent} extension from a concept into a reality. This would involve developing a prototype system and deploying it on the simulation platform to evaluate its performance in user roaming scenarios.
    \item \textit{Testing Under More Adverse Network Conditions}: The existing simulation platform could be extended to more complex network environment. By simulating a combination of delay, jitter, packet loss, and disruptions, a more comprehensive stress test could be conducted to evaluate the ultimate performance and robustness of DTN email proposals.
\end{itemize}
